\documentclass{amsart}

\usepackage{graphicx}
\usepackage{tikz}
\usepackage{ytableau}
\usepackage{float}
\usepackage[skip=2pt]{caption}
\usepackage{subfig}

\newtheorem{theorem}{Theorem}[section]
\newtheorem{lemma}[theorem]{Lemma}

\theoremstyle{definition}
\newtheorem{definition}[theorem]{Definition}
\newtheorem{example}[theorem]{Example}
\newtheorem{xca}[theorem]{Exercise}
\newtheorem{algorithm}[theorem]{Algorithm}

\theoremstyle{remark}
\newtheorem{remark}[theorem]{Remark}

\numberwithin{equation}{section}

%    Absolute value notation
\newcommand{\abs}[1]{\lvert#1\rvert}

%    Blank box placeholder for figures (to avoid requiring any
%    particular graphics capabilities for printing this document).
\newcommand{\blankbox}[2]{%
  \parbox{\columnwidth}{\centering
%    Set fboxsep to 0 so that the actual size of the box will match the
%    given measurements more closely.
    \setlength{\fboxsep}{0pt}%
    \fbox{\raisebox{0pt}[#2]{\hspace{#1}}}%
  }%
}

\setlength{\abovecaptionskip}{10pt plus 3pt minus 2pt}

\begin{document}

\title{PHYS 150 Term Paper - Young Tableaux and Its Applications}

\author{Yiluo Li}
\address{College of Creative Studies, University of California, Santa Barbara, CA 93106}
\email{yiluo\_li@ucsb.edu}

%    General info
\date{\today}
\keywords{Group Theory, Young Tableaux, Combinatorics, }

%\begin{abstract}
%\end{abstract}

\maketitle

\section{Introduction}
brief discussion on young tableaux's significance

\subsection{Partition and Young Diagram}
First of all, let us consider a partition $\lambda$ on a set of $n$ elements, which can be written as an ordered set of numbers $\{\lambda_1\geq\lambda_2\geq...\geq\lambda_m\geq 0\}$ and whose size $\abs{\lambda}\equiv\sum_i\lambda_i=n$. To visualize this partition, we can use a Young diagram, a set of left justified cells with $k$ rows, each $i$th row with $\lambda_i$ cells, where $\lambda_i$ is weakly decreasing as we go down the rows. An example of the partition $\{4, 2, 2, 1\}$ is shown below in \textbf{Figure \ref{fig:young_diagram}}.
\begin{figure}[H]
	\centering
	\ydiagram{4, 2, 2, 1}
	\caption{Young diagram for partition $\{4,2,2,1\}$\label{fig:young_diagram}}
\end{figure}

A Young tableau is a Young diagram with numbers within each cell. A semi-standard Young tableau is one whose number in the cells are weakly decreasing to the right to the bottom. A semi-standard tableau of size $\abs{\lambda}=n$ is considered standard if its filling is a bijective assignment from $\{1, 2, 3...n\}$. Example of both cases are shown below for partition $\{4, 2, 2, 1\}$.

\begin{figure}[H]
	\centering
	\subfloat[]{
		\begin{ytableau}
			1	&	2	&	2	&	3\\
			3	&	5\\
			5	&	7\\
			9
		\end{ytableau}
	}
	\qquad\qquad\qquad
	\subfloat[]{
		\begin{ytableau}
			1	&	3	&	5	&	7\\
			2	&	4\\
			6	&	9\\
			8
		\end{ytableau}
	}
	\caption{(A). Semi-standard Young tableau for partition $\{4, 2, 2, 1\}$; (B). Standard Young tableau for partition $\{4, 2, 2, 1\}$ with $\abs{\lambda}=n=9$}
\end{figure}

\begin{definition}
A hook $H(i, j)$ is a set of cells to the right or below the cell at $i$th row and $j$th column, plus the $(i, j)$ cell itself. 
\end{definition}

\begin{definition}
Hook length $h_\lambda(i, j)$ of a Young diagram of shape $\lambda$ is the number cells in the hook $H_\lambda(i, j)$.
\end{definition}

\begin{figure}[H]
	\centering
	\begin{ytableau}
		7	&	5	&	2	&	1\\
		4	&	2\\
		3	&	1\\
		1
	\end{ytableau}
	\caption{A tableaux of $\lambda=\{4,2,2,1\}$ filled by each cell's hook length\label{fig:tab_hook}}
\end{figure}

With the hook length defined, we can find the total number of standard tableaux of a given shape $\lambda$. There are a total of $n!$ ways to place the $\{1, 2, 3...n\}$ in each cell. Since the tableau is standard, we know that for each possible filling, the number in cell $(i, j)$ cannot be permuted to the right or below, thereby decreasing the multiplicity by $h_\lambda(i, j)$, and we have arrived at the hook length formula, giving the total number of standard tableaux $d_\lambda$ in shape $\lambda$:
\begin{equation}
	d_\lambda = \frac{\abs{\lambda}!}{\prod_{i, j}h_\lambda(i, j)}
\end{equation}

\newpage
\begin{lemma}
Let $f, g\in  A(X)$ and let $E$, $F$ be cozero
sets in $X$.
\begin{enumerate}
\item If $f$ is $E$-regular and $F\subseteq E$, then $f$ is $F$-regular.

\item If $f$ is $E$-regular and $F$-regular, then $f$ is $E\cup
F$-regular.

\item If $f(x)\ge c>0$ for all $x\in E$, then $f$ is $E$-regular.

\end{enumerate}
\end{lemma}

The following is an example of a proof.

\begin{proof} Set $j(\nu)=\max(I\backslash a(\nu))-1$. Then we have
\[
\sum_{i\notin a(\nu)}t_i\sim t_{j(\nu)+1}
  =\prod^{j(\nu)}_{j=0}(t_{j+1}/t_j).
\]
Hence we have
\begin{equation}
\begin{split}
\prod_\nu\biggl(\sum_{i\notin
  a(\nu)}t_i\biggr)^{\abs{a(\nu-1)}-\abs{a(\nu)}}
&\sim\prod_\nu\prod^{j(\nu)}_{j=0}
  (t_{j+1}/t_j)^{\abs{a(\nu-1)}-\abs{a(\nu)}}\\
&=\prod_{j\ge 0}(t_{j+1}/t_j)^{
  \sum_{j(\nu)\ge j}(\abs{a(\nu-1)}-\abs{a(\nu)})}.
\end{split}
\end{equation}
By definition, we have $a(\nu(j))\supset c(j)$. Hence, $\abs{c(j)}=n-j$
implies (5.4). If $c(j)\notin a$, $a(\nu(j))c(j)$ and hence
we have (5.5).
\end{proof}

\begin{quotation}
This is an example of an `extract'. The magnetization $M_0$ of the Ising
model is related to the local state probability $P(a):M_0=P(1)-P(-1)$.
The equivalences are shown in Table~\ref{eqtable}.
\end{quotation}

\begin{table}[ht]
\caption{}\label{eqtable}
\renewcommand\arraystretch{1.5}
\noindent\[
\begin{array}{|c|c|c|}
\hline
&{-\infty}&{+\infty}\\
\hline
{f_+(x,k)}&e^{\sqrt{-1}kx}+s_{12}(k)e^{-\sqrt{-1}kx}&s_{11}(k)e^
{\sqrt{-1}kx}\\
\hline
{f_-(x,k)}&s_{22}(k)e^{-\sqrt{-1}kx}&e^{-\sqrt{-1}kx}+s_{21}(k)e^{\sqrt
{-1}kx}\\
\hline
\end{array}
\]
\end{table}

\begin{definition}
This is an example of a `definition' element.
For $f\in A(X)$, we define
\begin{equation}
\mathcal{Z} (f)=\{E\in Z[X]: \text{$f$ is $E^c$-regular}\}.
\end{equation}
\end{definition}

\begin{remark}
This is an example of a `remark' element.
For $f\in A(X)$, we define
\begin{equation}
\mathcal{Z} (f)=\{E\in Z[X]: \text{$f$ is $E^c$-regular}\}.
\end{equation}
\end{remark}

\begin{example}
This is an example of an `example' element.
For $f\in A(X)$, we define
\begin{equation}
\mathcal{Z} (f)=\{E\in Z[X]: \text{$f$ is $E^c$-regular}\}.
\end{equation}
\end{example}

\begin{xca}
This is an example of the \texttt{xca} environment. This environment is
used for exercises which occur within a section.
\end{xca}

The following is an example of a numbered list.

\begin{enumerate}
\item First item.
In the case where in $G$ there is a sequence of subgroups
\[
G = G_0, G_1, G_2, \dots, G_k = e
\]
such that each is an invariant subgroup of $G_i$.

\item Second item.
Its action on an arbitrary element $X = \lambda^\alpha X_\alpha$ has the
form
\begin{equation}\label{eq:action}
[e^\alpha X_\alpha, X] = e^\alpha \lambda^\beta
[X_\alpha X_\beta] = e^\alpha c^\gamma_{\alpha \beta}
 \lambda^\beta X_\gamma,
\end{equation}

\begin{enumerate}
\item First subitem.
\[
- 2\psi_2(e) =  c_{\alpha \gamma}^\delta c_{\beta \delta}^\gamma
e^\alpha e^\beta.
\]

\item Second subitem.
\begin{enumerate}
\item First subsubitem.
In the case where in $G$ there is a sequence of subgroups
\[
G = G_0, G_1, G_2, \ldots, G_k = e
\]
such that each subgroup $G_{i+1}$ is an invariant subgroup of $G_i$ and
each quotient group $G_{i+1}/G_{i}$ is abelian, the group $G$ is called
\textit{solvable}.

\item Second subsubitem.
\end{enumerate}
\item Third subitem.
\end{enumerate}
\item Third item.
\end{enumerate}

Here is an example of a cite. See \cite{A}.

\begin{theorem}
This is an example of a theorem.
\end{theorem}

\begin{theorem}[Marcus Theorem]
This is an example of a theorem with a parenthetical note in the
heading.
\end{theorem}

\begin{figure}[tb]
%\blankbox{.6\columnwidth}{5pc}
\includegraphics{fig/lion}
\caption{This is an example of a figure caption with text.}
\label{firstfig}
\end{figure}

\begin{figure}[tb]
%\blankbox{.75\columnwidth}{3pc}
\includegraphics{fig/lion.png}
\caption{}\label{otherfig}
\end{figure}

\section{Some more list types}
This is an example of a bulleted list.

\begin{itemize}
\item $\mathcal{J}_g$ of dimension $3g-3$;
\item $\mathcal{E}^2_g=\{$Pryms of double covers of $C=\openbox$ with
normalization of $C$ hyperelliptic of genus $g-1\}$ of dimension $2g$;
\item $\mathcal{E}^2_{1,g-1}=\{$Pryms of double covers of
$C=\openbox^H_{P^1}$ with $H$ hyperelliptic of genus $g-2\}$ of
dimension $2g-1$;
\item $\mathcal{P}^2_{t,g-t}$ for $2\le t\le g/2=\{$Pryms of double
covers of $C=\openbox^{C'}_{C''}$ with $g(C')=t-1$ and $g(C'')=g-t-1\}$
of dimension $3g-4$.
\end{itemize}

This is an example of a `description' list.

\begin{description}
\item[Zero case] $\rho(\Phi) = \{0\}$.

\item[Rational case] $\rho(\Phi) \ne \{0\}$ and $\rho(\Phi)$ is
contained in a line through $0$ with rational slope.

\item[Irrational case] $\rho(\Phi) \ne \{0\}$ and $\rho(\Phi)$ is
contained in a line through $0$ with irrational slope.
\end{description}

\nocite{*}
\bibliography{ref}
\bibliographystyle{amsplain}

\end{document}

%------------------------------------------------------------------------------
% End of journal.tex
%------------------------------------------------------------------------------
